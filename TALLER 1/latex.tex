\documentclass[12pt]{report}
\usepackage[spanish]{babel}
\usepackage[latin9]{inputenc}
\usepackage{algorithm,algpseudocode}
\usepackage{amsfonts}

\title {
    {Taller 1}  \\
    {\large Pontificia Universidad Javeriana}\\
}
\author{Nicolas Lopez Fernandez }
\date{Agosto 1 2019}

\begin{document}

\maketitle

\section{Descripcion y formalizacion del problema}
El problema, informalmente, se define como encontrar el dato minimo y dato maximo de un arreglo rotado.

Asi, para el arreglo S=[67,89,90,23,25,34,50,60], el minimo es 23 y el maximo es 90.

Formalmente, se dice que: Dada una secuenca rotada $S$ de elementos $a_{i} \in \mathbb{T}$, donde se defina la relacion de orden parcial !!!!MENOR!!!! para el minimo y !!!!MAYOR!!!! para el maximo, informar los numeros $b$,$c \in$ $S$ donde $b$ cumple con la relacion de orden parcial !!!!MENOR!!!! y $c$ cumple con la relacion de orden parcial !!!!MAYOR!!!! de toda la secuencia $S$. Ahora, la definicion del contrato seria:
\begin{itemize}

\item \textbf{Entradas}: Una secuencia rotada $S$ de $n$
    numeros: $S=\left<a_{n-1},a_{n},a_{1},a_{2},\cdots,a_{n-2}\right>$ donde
    $a_{i}\in \mathbb{T}$
\item \textbf{Salidas}: Dos resultados $b=a_{1}$ y $c=a_{n}$ donde $b$ y $c$ son el menor y       mayor dato de la secuencia rotada $S$

\end{itemize}

\section{Fuerza bruta}
\begin{algorithm}[H]
  \begin{algorithmic}[1]
    \Procedure{MaxMinRotadoBF}{$S$}
      \State$maxi \leftarrow 0$
      \State$mini \leftarrow 0$
      \For{$i \leftarrow 1$ \textbf{to} $\left|S\right|$}
        \If{$S[i] \ge maxi$}
            \State$maxi \leftarrow S[i]$
        \EndIf
      \EndFor
      \State$mini \leftarrow maxi$
      \For{$i \leftarrow 1$ \textbf{to} $\left |S\right|$}
        \If{$S[i] \le mini$}
            \State$mini \leftarrow S[i]$
        \EndIf
      \EndFor
      \State\Return $[mini, maxi]$
    \EndProcedure
  \end{algorithmic}
  \caption{Algoritmo maximo y minimo por fuerza bruta.}
\end{algorithm}

\section{Dividir y vencer}
\begin{algorithm}[H]
  \begin{algorithmic}[2]
    \Procedure{MMRotado}{$S,l,h$}
        \If{$h \le l$}
            \State\Return $[0,-1]$
        \ElsIf{$h==l$}
            \State\Return $[S[l],S[h]]$
        \ElsIf{$h==1$}
            \State\Return $[S[0],S[1]]$
        \ElsIf{$S[0]>S[1]$}
            \State\Return $[S[0],S[1]]$
        \Else
            \If{$h==2$}
                \State $S.append(\left |S\right| + 1)$
                \State $h \leftarrow h + 1$
            \EndIf
            \State $mitad \leftarrow\floor{(h+2)/2}$
            \If{$S[mitad] > S[mitad-1]$}
                \State $aux \leftarrow S[mitad:\left |S\right|]$
                \State\Return $MMRotado(aux,0,\left |S\right|)$
            \EndIf
            \If{$S[mitad] < S[mitad+1]$}
                \State $aux = S[0:mitad+1]$
                \State\Return $MMRotado(aux,0,mitad)$
            \EndIf
        \EndIf
    \EndProcedure
  \end{algorithmic}
  \caption{Algoritmo Maximo y minimo por dividir y vencer.}
\end{algorithm}

\begin{algorithm}[H]
  \begin{algorithmic}[3]
    \Procedure{MaxMinRotadoDV}{$S$}
        \state\Return $MMRotado(S,0,\left|S\right|)$
    \EndProcedure
  \end{algorithmic}
  \caption{Algoritmo Maximo y minimo por dividir y vencer.}
\end{algorithm}

\section{Invariantes}
El algoritmo de ''Fuerza Bruta'' tiene como invariante: En cada ciclo de busqueda se encuentra o el maximo o el minimo

El algoritmo de ''Dividir-y-vender'' tiene como invariante: El menor y el mayor siempre esta en la secuencia que se esta dividiendo

\section{Analisis de complejidad}
En el algoritmo de ''Fuerza Bruta'', resulta evidente por inspeccion de codigo que su orden de complejidad es $0(n^3)$

Para la version ''Dividir-y-vencer'', se tiene la ecuacion de recurrencia:

\begin{equation}
T(n)=
\left\{
    \begin{array}{rcl}
        O(1) &; &b \ge e\\
        T\left(\frac{n}{2}\right)+O(n) & ; & b<e
    \end{array}
\right.;
\end{equation}
que tiene un orden de complejidad $\Theta(\log n)$, despues de usar el teorema maestro.
\end{document}
