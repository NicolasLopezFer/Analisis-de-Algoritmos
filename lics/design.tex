%% =========================================================================
%% @author Leonardo Florez-Valencia (florez-l@javeriana.edu.co)
%% =========================================================================

%% == Includes
\documentclass{article}
\usepackage[spanish]{babel}
\usepackage[latin9]{inputenc}
\usepackage{algorithm,algpseudocode}
\usepackage{amsfonts}
\usepackage{mathtools}

%% == Title
\title{Subsecuencia contigua creciente m�s larga}
\author{Leonardo Fl�rez-Valencia}
\date{2014-2019}

%% == Main document
\begin{document}

\maketitle

\section{Descripci�n y formalizaci�n del problema}

El problema, informalmente, se define como: encontrar el sub-arreglo creciente
m�s largo de un arreglo.

As�, para el arreglo $S=\left[ 9, 7, 7, 1, 2, 3, 4, 6, 5 \right]$ el sub-arreglo m�s
largo ser�a $LICS = \left[ 1, 2, 3, 4, 6 \right]$.

Formalmente, se dice que: Dada una secuencia $S$
de elementos $a_{i} \in \mathbb{T}$, donde se define la relaci�n de
orden total $<$, informar la secuencia $L \in S$
donde los elementos contiguos cumplan la relaci�n de orden total
$\le$.
Ahora, la definici�n del contrato ser�a:
\begin{itemize}
\item \textbf{Entradas}: Una secuencia $S$ de $n$
  n�meros: $S=\left<a_{1},a_{2},\cdots,a_{n}\right>$ donde
  $a_{i}\in\mathbb{T}$ y en $\mathbb{T}$ est� definida la relaci�n
  de orden total $\le$.
\item \textbf{Salidas}: Una subsecuencia $L=\left<a_{i},\cdots,a_{j}\right>\mid a_{i}\le a_{i+1}\le\cdots\le a_{j-1} \land L \in S$
\end{itemize}

\section{Fuerza bruta}
\begin{algorithm}[H]
  \begin{algorithmic}[1]
    \Procedure{LICS}{$S$}
      \State$maxS \leftarrow -\infty$
      \State$maxI \leftarrow 0$
      \State$maxJ \leftarrow 0$
      \For{$i \leftarrow 1$ \textbf{to} $\left|S\right|$}
        \For{$j \leftarrow i$ \textbf{to} $\left|S\right|$}
          \State$isOrdered \leftarrow True$
          \For{$k \leftarrow i+1$ \textbf{to} $j$}
            \If{$S[k] \le S[k-1]$}
              \State $isOrdered \leftarrow False$
            \EndIf
          \EndFor
          \If{$isOrdered \land maxS < \left(j - i\right)$}
            \State$maxS \leftarrow j - i$
            \State$maxI \leftarrow i$
            \State$maxJ \leftarrow j$
          \EndIf
        \EndFor
      \EndFor
      \State\Return $[maxI, maxJ]$
    \EndProcedure
  \end{algorithmic}
  \caption{Algoritmo de LICS fuerza bruta.}
\end{algorithm}

Este algoritmo presenta una cota superior de $O(n^3)$, calculada por simple
inspecci�n de c�digo.

\section{Dividir y vencer}
\begin{algorithm}[H]
  \begin{algorithmic}[1]
    \Procedure{LICSAux}{$S,b,e$}
      \If{$b = e$}
        \State\Return$[b, b]$
      \ElsIf{$e<b$}
        \State\Return$[0, -1]$
      \Else
        \State$q \leftarrow \lfloor (b+e) \div 2 \rfloor$
        \State$[L_i, L_j] \leftarrow LICSAux\left( S, b, q-1 \right)$
        \State$[R_i, R_j] \leftarrow LICSAux\left( S, q+1, e \right)$
        \State$[C_i, C_j] \leftarrow LICSPivot\left( S, b,q, e \right)$
        \If{$(L_j - L_i)>(R_j - R_i) \land (L_j - L_i)>(C_j - C_i)$}
          \State\Return$[L_i, L_j]$
        \ElsIf{$(R_j - R_i)>(L_j - L_i) \land (R_j - R_i)>(C_j - C_i)$}
          \State\Return$[R_i, R_j]$
        \Else
          \State\Return$[C_i, C_j]$
        \EndIf
      \EndIf
    \EndProcedure
  \end{algorithmic}
  \caption{Algoritmo de LICS dividir y vencer.}
\end{algorithm}

\begin{algorithm}[H]
  \begin{algorithmic}[1]
    \Procedure{LICSPivot}{$S,b,q,e$}
      \State$i \leftarrow q$
      \While{$b < i \land S[i-1] \le S[i]$}
        \State$i \leftarrow i - 1$
      \EndWhile
      \State$j \leftarrow q$
      \While{$j < e \land S[j] \le S[j+1]$}
        \State$j \leftarrow j + 1$
      \EndWhile
      \State\Return$[i, j]$
    \EndProcedure
  \end{algorithmic}
  \caption{Algoritmo de LICS dividir y vencer: funci�n del pivote.}
\end{algorithm}

\begin{algorithm}[H]
  \begin{algorithmic}[1]
    \Procedure{LICS}{$S$}
      \State\Return$LICSAux(S,1,|S|)$
    \EndProcedure
  \end{algorithmic}
  \caption{Algoritmo de LICS dividir y vencer: funci�n principal.}
\end{algorithm}

\section{Invariantes}

El algoritmo de ``fuerza bruta'' tiene como invariante:

El algoritmo ``dividir-y-vencer'' tiene como invariante:

\section{An�lisis de complejidad}

En el caso del algoritmo de ``fuerza bruta'', resulta evidente por inspecci�n de
c�digo que su orden de complejidad es $O(n^3)$.

Para la versi�n ``dividir-y-vencer'', se tiene la ecuaci�n de recurrencia:

\begin{equation}
T(n)=
\left\{
  \begin{array}{rcl}
    O(1) & ; & b\ge e\\
    2T\left(\frac{n}{2}\right)+O(n) & ; & b<e
  \end{array}
\right.;
\end{equation}
que tiene un orden de complejidad $\Theta(n \log n)$, despu�s de usar el teorema maestro.


\end{document}



%% eof - $RCSfile$
