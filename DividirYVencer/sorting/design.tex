%% =========================================================================
%% @author Leonardo Florez-Valencia (florez-l@javeriana.edu.co)
%% =========================================================================

%% == Includes
\documentclass{article}
\usepackage[spanish]{babel}
\usepackage[latin9]{inputenc}
\usepackage{algorithm,algpseudocode}
\usepackage{amsfonts}

%% == Title
\title{An�lisis del problema de ordenamiento de secuencias}
\author{Leonardo Fl�rez-Valencia}
\date{2014-2019}

%% == Main document
\begin{document}

\maketitle

\section{Descripci�n y formalizaci�n del problema}

El problema, informalmente, se define como: Ordenar una
``lista'' / ``arreglo'' / ``vector'' / ``conjunto'' / ``mont�n'' de n�meros.

Formalmente, se dice que: Dada una secuencia $S$
de elementos $a_{i} \in \mathbb{T}$, donde se define la relaci�n de
orden parcial $\le$, producir una nueva secuencia $S'$
donde los elementos contiguos cumplan la relaci�n de orden parcial
$\le$.
Ahora, la definici�n del contrato ser�a:
\begin{itemize}
\item \textbf{Entradas}: Una secuencia $S$ de $n$
  n�meros: $S=\left<a_{1},a_{2},\cdots,a_{n}\right>$ donde
  $a_{i}\in\mathbb{T}$ y en $\mathbb{T}$ est� definida la relaci�n
  de orden parcial $\le$.
\item \textbf{Salidas}: Una permutaci�n $S'=\left<a'_{1},a'_{2},\cdots,a'_{n}\right>\mid a'_{1}\le a'_{2}\le\cdots\le a'_{n}\land a_{i}'\in S\forall i$
\end{itemize}

\section{Ordenamiento burbuja}
\begin{algorithm}[H]
  \begin{algorithmic}[1]
    \Procedure{BubbleSort}{$S$}
      \For{$i \leftarrow 1$ \textbf{to} $\left|S\right|$}
        \For{$j \leftarrow 1$ \textbf{to} $\left|S\right|-i$}
          \If{$S\left[j+1\right]<S\left[j\right]$}
            \State$aux \leftarrow S\left[j\right]$
            \State$S\left[j\right] \leftarrow S\left[j+1\right]$
            \State$S\left[j+1\right] \leftarrow aux$
          \EndIf
        \EndFor
      \EndFor
    \EndProcedure
  \end{algorithmic}
  \caption{Algoritmo de ordenamiento por burbuja.}
\end{algorithm}

\section{Ordenamiento por inserci�n}
\begin{algorithm}[H]
  \begin{algorithmic}[1]
    \Procedure{InsertionSort}{$S$}
      \For{$j\leftarrow2$ \textbf{to} $\left|S\right|$}
        \State$k\leftarrow S\left[j\right]$
        \State$i\leftarrow j-1$
        \While{$0<i\land k<S\left[i\right]$}
          \State$S\left[i+1\right]\leftarrow S\left[i\right]$
          \State$i\leftarrow i-1$
        \EndWhile
        \State$S\left[i+1\right]\leftarrow k$
      \EndFor
    \EndProcedure
  \end{algorithmic}
  \caption{Algoritmo de ordenamiento por inserci�n.}
\end{algorithm}

\end{document}

%% eof - $RCSfile$
